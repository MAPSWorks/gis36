\documentclass[12pt,a4paper,oneside]{article} %команды, относящиеся ко всему документу и устанавливающие различные параметры оформления текста
\usepackage[T2A]{fontenc} % команда, в аргументе которой стоит (через запятую) список подключаемых этой командой стилевых пакетов
\usepackage[utf8]{inputenc}
\usepackage[english,russian]{babel}
\frenchspacing
\usepackage[pdftex]{graphicx}
\usepackage{setspace}
\usepackage[usenames,dvipsnames]{color}
\usepackage{hhline}
\usepackage[14pt]{extsizes}
\usepackage{fancyhdr} % пакет для установки колонтитулов

% Левое поле 3 cm
\oddsidemargin=0mm
\hoffset=4.6mm

% Верхнее поле 2 cm
\topmargin=0pt
\voffset=-5.4mm
\headheight=0mm
\headsep=1cm

\textwidth=163mm
\textheight=245mm
\parindent=1.25cm
\tabcolsep=1mm
\itemsep=0pt

\begin{document}
\onehalfspacing
\pagestyle{empty} % нумерация выкл.
%Н:задание на уир и кп
\newpage
\begin{center}
{\small МИНИСТЕРСТВО ОБРАЗОВАНИЯ И НАУКИ РОССИЙСКОЙ ФЕДЕРАЦИИ

Федеральное государственное бюджетное образовательное учреждение высшего профессионального образования

\textbf{"Национальный исследовательский ядерный университет "МИФИ"}
\\[50pt]
}
\textbf{Факультет Кибернетики}

\textbf{Кафедра № 36 (Информационные технологии)
\\[50pt]
\Large{Задание на УИР и КП}}
\begin{flushleft}

Студентке группы К7-361 Лаврентьевой Марине Олеговне
\\[40pt]
\end{flushleft}

\textbf{ТЕМА УИР и КП}

Разработка модуля преобразования геоданных из формата OpenStreetMap в формат OpenGIS
\\[20pt]
\textbf{ЗАДАНИЕ}
\end{center}
\begin{enumerate}
\item Изучение структуры xml-документа с геоданными в формате OpenStreetMap.
\item Построение XPath-запросов для извлечения геоданных об объектах заданного типа.
\item Изучение стандарта OpenGIS и разработка структуры базы данных для хранения геоданных в формате OpenGIS.
\item Построение SQL-запросов для сохранения и извлечения геоданных в формате OpenGIS.
\item Реализация модуля преобразования геоданных из формата OpenStreetMap в формат OpenGIS.
\end{enumerate}

\newpage
\begin{flushleft}
Место выполнения УИР и КП \underline{\hspace{9cm}}

Руководитель \underline{\hspace{9cm} (Муравьёв С.К.)}

Дата выдачи задания \underline{\hspace{11cm}}
\end{flushleft}

\newpage
\begin{center}
\textbf{ОТЗЫВ О РАБОТЕ СТУДЕНТА}
\end{center}
\vfill
Руководитель \underline{\hspace{6cm}\glqq{\hspace{1cm}}\grqq\hspace{4cm}2011г.}
%К: задание на уир и кп
\end{document}
