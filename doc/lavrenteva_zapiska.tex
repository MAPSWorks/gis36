\documentclass[12pt,a4paper,oneside]{article} %команды, относящиеся ко всему документу и устанавливающие различные параметры оформления текста
\usepackage[T2A]{fontenc} % команда, в аргументе которой стоит (через запятую) список подключаемых этой командой стилевых пакетов
\usepackage[utf8]{inputenc}
\usepackage[english,russian]{babel}
\frenchspacing
\usepackage[pdftex]{graphicx}
\usepackage{setspace}
\usepackage[usenames,dvipsnames]{color}
\usepackage{hhline}
\usepackage[14pt]{extsizes}
\usepackage{fancyhdr} % пакет для установки колонтитулов

% Левое поле 3 cm
\oddsidemargin=0mm
\hoffset=4.6mm

% Верхнее поле 2 cm
\topmargin=0pt
\voffset=-5.4mm
\headheight=0mm
\headsep=1cm

\textwidth=163mm
\textheight=245mm
\parindent=1.25cm
\tabcolsep=1mm
\itemsep=0pt

\begin{document}
\onehalfspacing
\pagestyle{empty} % нумерация выкл.
%Н: титульный лист
\newpage
\begin{center}
{\small МИНИСТЕРСТВО ОБРАЗОВАНИЯ И НАУКИ РОССИЙСКОЙ ФЕДЕРАЦИИ

Федеральное государственное бюджетное образовательное учреждение высшего профессионального образования

\textbf{"Национальный исследовательский ядерный университет "МИФИ"}
\\[50pt]
}
\textbf{Факультет Кибернетики}

\textbf{Кафедра № 36 (Информационные технологии)
\\[50pt]
\Large{Пояснительная записка}}

к курсовому проекту на тему:

\textbf{Разработка модуля преобразования геоданных из формата OpenStreetMap в формат OpenGIS}
\\[60pt]
\end{center}
\begin{flushleft}
Группа К7-361

Студент \underline{\hspace{9,5cm} (Лаврентьева М.О.)}

Руководитель \underline{\hspace{9cm} (Муравьёв С.К.)}

Оценка \underline{\hspace{14,25cm}}

Члены комиссии \underline{\hspace{12,2cm}}

\hspace{3,7cm} \underline{\hspace{12,2cm}}

\hspace{3,7cm} \underline{\hspace{12,2cm}}

\hspace{3,7cm} \underline{\hspace{12,2cm}}

\end{flushleft}
\begin{center}
\vfill
Москва 2011
\end{center}
%K: титульный лист
%Н: литература
\newpage
\begin{center}
\textbf{Литература}
\end{center}
%К: литература
%Н: оглавление
\newpage
\begin{center}
\textbf{Оглавление}
\end{center}
%К: оглавление
%Н: введение
\newpage
\begin{center}
\textbf{Введение}
\end{center}

В настоящее время широкое распространение в информационных технологиях получили геоинформационные системы (ГИС). ГИС представляет собой аппаратно-программный человеко-машинный комплекс,\\ обеспечивающий сбор, обработку, отображение и распространение геоданных.

Понятие геоданные (пространственные данные, географические данные) включает в себя цифровые данные о пространственных объектах, сведения об их местоположении и свойствах, пространственных и непространственных атрибутах.

ГИС позволяет наиболее эффективно и комплексно  использовать совершенно различные геоданные при решении научных и прикладных географических задач, связанных с инвентаризацией, анализом, моделированием, прогнозированием и управлением окружающей средой и территориальной организацией общества. Термин ГИС в используется в более узком смысле — как инструмент (программный продукт), позволяющий пользователям искать, анализировать и редактировать цифровые карты, а также получать различную дополнительную информацию. Наиболее известные и часто используемые современные ГИС - это Google Планета Земля, Yandex-пробки.

В своей учебно-исследовательской работе я рассмотрю задачу сбора, обмена и хранения геоданных. Картографические данные будут получены из открытого источника в интернете OpenStreetMap.org в формате xml.

Однако обмен и хранение пространственных данных требует развитой и всеобъемлющей системы стандартов представления геоданных. Такой системой стандартов для многих приложений на данный момент является формат Open Geospatial Consortium (OGC) (или сокращенно OpenGIS).  Таким образом  для дальнейшего анализа и представления полученных геоданных их необходимо представить в данном формате \\ (OpenGIS).

Итак, \underline{цель работы}: описание алгоритма работы и реализация функционала модуля преобразования геоданных из формата OpenStreetMap в формат OpenGIS.


%К: введение
\end{document}
